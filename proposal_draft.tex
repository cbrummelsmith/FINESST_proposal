
  
\documentclass[a4paper, 12pt]{article}

%\usepackage{microtype}
\usepackage{natbib}
\usepackage{hyperref}
\usepackage{graphicx}
\usepackage{booktabs}
\usepackage{xspace}
\usepackage{paralist}
\usepackage{amsmath}
\usepackage[dvipsnames]{xcolor}
\usepackage[final]{pdfpages}
\usepackage{subfig}


%\usepackage{ragged2e}
%\usepackage[format=plain,justification=RaggedRight,font=footnotesize]{caption}

\usepackage[style=plain,floatrowsep=qquad]{floatrow}

%\setlength{\parskip}{\baselineskip}
%\setlength{\parindent}{0em}

%\usepackage{sectsty}
%\allsectionsfont{\raggedright\bfseries\large}
%\subsectionfont{\raggedright\normalfont\itshape\normalsize}

\usepackage[compact]{titlesec}

\usepackage{aas_macros}
\usepackage{amsmath}

\usepackage[charter]{mathdesign}

\usepackage[margin=1in]{geometry}

\usepackage{fancyhdr}
\pagestyle{fancyplain}
\renewcommand{\headrulewidth}{0pt}
\lhead{}
\chead{}
\rhead{}
\lfoot{}
\cfoot{\thepage}
\rfoot{}

\hypersetup{pdfpagemode={UseOutlines},
  bookmarksopen,
  colorlinks,
  linkcolor={Black},
  citecolor={Black},
  urlcolor={RoyalBlue}}

\bibliographystyle{apj}

\newcommand*{\eg}{{e.\,g.,}\xspace}
\newcommand*{\ie}{{i.\,e.,}\xspace}

\newcommand{\todo}[1]{{\color{red} TODO: #1}}



%%%%%%%%%%%%%%%%%%%%%%%%%%%%%%%%%%%%%%%%%%%%%%%%%%%%%%%%%%%%%%%%%%%%%%%%%%%%%%%%
%% fix a bug in bibtex: only the year in author-year should be a link
%%
\RequirePackage{etoolbox}
\makeatletter

% Patch case where name and year have no delimiter
\patchcmd{\NAT@citex}
  {\@citea\NAT@hyper@{\NAT@nmfmt{\NAT@nm}\NAT@date}}
  {\@citea\NAT@nmfmt{\NAT@nm}\NAT@hyper@{\NAT@date}}
  {}% Do nothing if patch works
  {}% Do nothing if patch fails

% Patch case where name and year have basic delimiter
\patchcmd{\NAT@citex}
  {\@citea\NAT@hyper@{%
     \NAT@nmfmt{\NAT@nm}%
     \hyper@natlinkbreak{\NAT@aysep\NAT@spacechar}{\@citeb\@extra@b@citeb}%
     \NAT@date}}
  {\@citea\NAT@nmfmt{\NAT@nm}%
   \NAT@aysep\NAT@spacechar%
   \NAT@hyper@{\NAT@date}}
  {}% Do nothing if patch works
  {}% Do nothing if patch fails

% Patch case where name and year are separated by a prenote
\patchcmd{\NAT@citex}
  {\@citea\NAT@hyper@{%
     \NAT@nmfmt{\NAT@nm}%
     \hyper@natlinkbreak{\NAT@spacechar\NAT@@open\if*#1*\else#1\NAT@spacechar\fi}%
       {\@citeb\@extra@b@citeb}%
     \NAT@date}}
  {\@citea\NAT@nmfmt{\NAT@nm}%
   \NAT@spacechar\NAT@@open\if*#1*\else#1\NAT@spacechar\fi%
   \NAT@hyper@{\NAT@date}}
  {}% Do nothing if patch works
  {}% Do nothing if patch fails

\makeatother
%% end bugfix
%%%%%%%%%%%%%%%%%%%%%%%%%%%%%%%%%%%%%%%%%%%%%%%%%%%%%%%%%%%%%%%%%%%%%%%%%%%%%%%%

% Alter some LaTeX defaults for better treatment of figures:
% See p.105 of "TeX Unbound" for suggested values.
% See pp. 199-200 of Lamport's "LaTeX" book for details.
% General parameters, for ALL pages:
\renewcommand{\topfraction}{0.9}    % max fraction of floats at top
\renewcommand{\bottomfraction}{0.8} % max fraction of floats at bottom
%
% Parameters for TEXT pages (not float pages):
\setcounter{topnumber}{2}
\setcounter{bottomnumber}{2}
\setcounter{totalnumber}{4}             % 2 may work better
\setcounter{dbltopnumber}{2}            % for 2-column pages
\renewcommand{\dbltopfraction}{0.9} % fit big float above 2-col. text
\renewcommand{\textfraction}{0.07}  % allow minimal text w. figs
%
% Parameters for FLOAT pages (not text pages):
\renewcommand{\floatpagefraction}{0.7}  % require fuller float pages
%
% N.B.: floatpagefraction MUST be less than topfraction !!
\renewcommand{\dblfloatpagefraction}{0.7} % require fuller float pages


\begin{document}
\fontsize{12}{16}\selectfont
\author{Corey Brummel-Smith}
\title{Title}
%% title
%\maketitle



\section{Introduction}

The elements that make up everything we see around us, in fact, nearly everything from the second row of the periodic table down, didn't exist before the formation of the formation of the first stars. Understanding the first stars is not just important for for understanding our cosmic origins; they're crucial in understanding the formation of large scale structures, and the evolution of the early universe as a whole. The difficulty is, even when we look deep into sky, far back in time, all these first stars are too faint or already dead, so studying them observationally is not currently possible. The only connection we have to them are the clues they've imprinted on the second generation of stars, by enriching them with their metals. When these, so called, Population III (Pop III) stars explode, they enrich the environment with metals formed in their supernovae, and the remnants can later collapse to form new, metal-poor stars \cite{...}. Some of these second generation (Pop II) stars have much longer lives and may still exist today in places like the galactic halo and ultra-faint dwarf galaxies (UFDGs) \citep{...}. One of the goals of stellar archeology is to infer the properties of the first stars by observing the distribution and metallicities of their long-lived, metal enriched descendants. However, since individual Pop II stars may be enriched by multiple Pop III progenitors, it is often impossible to infer the precise masses and metallicities of the first stars. We will address this problem with two separate types of numerical simulations. In the first project, we will investigate whether supernova induced star formation can produce second generation, metal-poor stars enriched by a single progenitor. In the second project, we will track metals from each individual Pop III star in a cosmological simulation, and determine exactly how many Pop III stars enrich the second generation stars. \textbf{Ultimately, this work will provide a deeper understanding of the transition away from a metal-free universe, and allow us to constrain the masses, metallicities, and distribution of the first stars.}

\section{Carbon Enhanced Metal-Poor Stars}

In some cases, the metallicity of the second stars is set entirely by the yields of the Pop III stars. The stellar life-cycle produces more metals with each successive generation, and thus, today, we see stars with a metal abundance orders of magnitude larger than that of the stars in the early universe. Since the metal abundance of the universe continually grows over time, metallicity is a good indicator of the age of a system. This is why low metallicity stars are the best places to study the chemical composition of the early universe and the first stars. Because of their low metallicity and old age, carbon-enhanced metal-poor (CEMP) stars are of particular interest in stellar archeology. These stars are divided into three groups based on their iron and carbon abundances. Group II and III are comprised of CEMP-no stars, which are CEMP stars with no overabundance of s-process elements \citep{Maeder2015}. 

CEMP-no stars are of particular interest to us because detailed analysis of their chemical composition strongly supports the hypothesis that the pre-stellar clouds in which these stars formed, were enriched by the first-generation stars. \citep{Yoon2016}. \cite{Yoon2016} also point out that more than one class of progenitors for group II and group III stars exist, and yet-to-be suggested sites of star formation and metal enrichment may account for the contrasts between Group II and III. \textbf{In our first project, we will investigate a new formation channel, fueled by triggered star formation, that could help explain the differences between Group II and Group III CEMP-no stars.}


\section{Proposed Work}
The overall goal of our two projects will be to use numerical simulations to track metals from individual population III stars from their supernova remnants, all the way to their final resting place in the stellar nurseries of the second generation stars. These simulations will provide a quantitative connection between the first stars and their metal poor descendants, which will allow us to determine the properties of real Pop III stars, from observations of CEMP stars and the metal poor galaxies in which they live. We accomplish this with two projects which each have their own additional aims and benefits. In project 1, we will use cosmological simulations to study how multiple Pop III progenitors seed the formation of the second stars, and test different initial mass functions (IMFs) for the Pop III stars. In project 2, we will perform supernova induced star formation simulations to determine the properties of stars that form from a single Pop III progenitor.


\subsection{Tracking Metals from Pop III to Pop II Stars in Cosmological Simulations}
\label{subsec:tracer_particles}

With this project, we will, for the first time, provide a direct connection between multiple progenitors and their metal-poor descendants by tracking the metals from the supernovae of individual population III stars. Not only will this allow us to report \textit{exactly} how many Pop III stars enrich a given Pop II star cluster, we will also know the relative fractions that come from each progenitor. Since we will be able to tell the exact mass and metal yields of each Pop III star, we will be able to make predictions about the properties of the progenitors of the actual CEMP-no stars observed in the halo of our galaxy and in UFDGs. 

To model the star formation process in a realistic, self-consistent manner, we will run a cosmological simulation initialized at a from a time before the formation of the first stars. The initial conditions for this simulation will be generated using MUSIC (Multi-scale initial conditions for cosmological simulations) \citep{Hahn2011}, using the most recent cosmological parameters obtained by \cite{Planck2018}. MUSIC allows us to first simulate a large, representative, volume of the universe, and then restart the simulation, "zooming-in" on a particular sub volume to more accurately resolve the star formation processs. This simulation will be run with Enzo: an adaptive mesh refinement (AMR) code for astrophysics \citep{Bryan2014}. The benefit of an AMR code is we can focus the bulk of the computational resources on regions on high-density high density dark matter halos and star-forming clouds. Enzo is also an ideal code to use for this simulation because it can model the effects of stellar radiation feedback, and radiative cooling; processes that are important to accurately model star formation \citep{...}. Enzo is also integrated with the Grackle chemical network solver \citep{...}, which will allow us to model the formation and destruction of $H_2$, as well many other chemical species, and track the ionization states of various elements. All of these affect the heating and cooling rates of the primordial gas, and are therefore important to properly model star formation \citep{...}.

In a cosmological simulation, resolving objects the size of individual stars is not computational tenable. We therefore use a sub-grid model for stars and stellar feedback. Stars in the simulation are represented as point particles, advected in a Lagrangian manner, in contrast to the advection of the fluid fields which is done on an Eulerian grid. An important feature of these star particles is they inject energy and momentum back into the grid via heat and radiation pressure. The stars also produce ionizing Lyman-Werner radiation capable of dissociating $H_2$ molecules \citep{...}. The stars will continue their feedback until they've exhausted their time on the main sequence, at which point they produce a supernova. Unlike previous simulations of Pop III stars, we will vary the explosion energy depending on the mass of the progenitor. This allows us to simulate different type's of explosions from faint supernovae to hypernovae. The lifetime of the Pop III stars will be determined from the \cite{Schaerer2002} and the supernova metal yields will be taken from \cite{Nomoto2006}.

When the Pop III stars explode, energy, momentum, and metals are deposited onto the grid in a spherical region around the star. However, unlike any previous cosmology simulation involving Pop III stars, we will inject tracer particles into the supernovae. These massless tracer particles will have identifiers unique to the Pop III star that produced them. They will be advected with with the metals from the supernovae, and thus, will track the proliferation of the metals from their source. Over time, these particles will collapse with the metals into high density regions where the second generation stars will form. Due to the unique identifiers of the tracer particles, \textbf{we will be able to tell exactly where the metals seeding the second-generation stars come from, and how many progenitors they have.} A qualitative illustration of this is shown in figure \ref{fig:tracer}. This is of great importance because it will provide a direct connection between the first and second stars. With a large enough sample of Pop III stars in the simulation, we will be able to make statistically significant conclusions about how often metal-poor stars are enriched by one or more progenitors and the fraction of metals coming from each. To ensure we have a sufficient number of dark matter halos and Pop III stars, we will choose a box size of 1 co-moving Mpc, a root grid resolution of $256^3$, with a maximum of 12 levels of refinement. This gives a minimum cell size of 0.1 pc. Given this simulation domain, we would expect on the order of 500 Pop III stars to form. For the sake of argument, if one assumes a second-generation star forming region has, at most, 5 distinct progenitors, this leaves a minimum of 100 individual sites to investigate the formation of metal-poor stars.

Not only will these simulations allow us to study the number of progenitors seeding the second stars, it will also allow us to determine where the progenitors lived. Pop II stars are either enriched internally, by progenitors within the same halo \citep{Chiaki2019}; or externally by progenitors in a neighboring halo \citep{Smith2015}. Our tracer particles will show which of these channels occurs more frequently, and the differences in the metal enrichment between these two scenarios. 


\begin{figure}
  \includegraphics[width=0.5\textwidth]{figures/tracer_final}
  \caption{Qualitative illustration of metal tracer particles in a cosmological simulation. Black and white colors represent gas density in an example cosmological simulation. Blue, pink, and green highlighted regions represent metal enrichment from the supernova remnants of the progenitor Pop III stars (blue, pink and green stars). Colored points represent tracer particles. Gold stars represent potential sites of second generation star formation. No stars are actually present in the example simulation. This is just an illustration to convey the idea of this project.}
  \label{fig:tracer}    
\end{figure}


\subsection{Supernova Induced Formation of the Second Stars}

We know from simulations, a typical formation chanel for the second-generation stars is the internal enrichment (IE) scenario \citep{Chiaki2019}. The IE scenario occurs when a Pop III star explodes, enriching the host dark matter halo with metals. The metal enriched supernova remnant eventually re-collapses to form new stars. This process occurs on a timescale of several tens of Myr \citep{Chiaki2019}, which is much longer than the lives of massive Pop III stars. For reference, a 40 $M_\odot$ Pop III star has a main sequence lifetime of 3.86 Myr \citep{Schaerer2002}, much shorter than the timescale of supernova remnant re-collapse. For this reason, it is possible and -- maybe even likely -- that an individual Pop II star will be seeded by multiple Pop III progenitors. In these cases, there are multiple interpretations or inferences one could make about the progenitors. For instance, a metal enriched star could have received its metals from a single, massive, Pop III star with high yields, or multiple less massive stars with smaller yields. We will investigating a new formation channel where this ambiguity is unlikely to be a problem, thereby providing a direct link between the properties of single Pop III stars and their metal-poor descendants. This can be accomplished through triggered star formation. This could happen when a Pop III star explodes in close proximity to a molecular cloud, mechanically crushing it into gravitational collapse, or by heating the surrounding environment enough for the increased pressure to induce gravitational collapse. A diagram of this is shown in figure \ref{fig:tsf}. If the formation a second generation star is induced by the supernova from a first generation Pop III star, the descendant is less likely to be contaminated by metals from other Pop III progenitors. This is due to the timescales involved. In the triggered star formation scenario, after the supernova begins to crush a nearby molecular cloud into gravitational collapse, stars will form in roughly the free-fall time of the cloud. This would typically be on the order of millions of years rather than several tens of millions. Therefore, the protostellar molecular cloud is less likely to be contaminated by metals from a separate Pop III stars. 

We will use numerical hydrodynamic simulations with radiative feedback to perform a detailed study of supernova induced star formation. We will run these simulations with Enzo for the reasons described in section \ref{subsec:tracer_particles}. By systematically studying the triggered star formation mechanism, we can determine the conditions necessary for this to occur in the early universe, and most importantly, identify a causal link between the the mass and yields of the progenitor Pop III star and the observable properties of the metal-poor descendants. To create this model, we will run many realizations of similar simulations spanning a wide parameter space. Our parameter space will include the mass of the progenitor, the explosion energy, the mass of the the molecular cloud, and the distance between the cloud and progenitor. By simulating a large parameter space, we can determine the optimal conditions that would allow this formation channel to occur. From these results, we will develop a model that will allow one to infer the properties of the progenitor, given the properties of the descendants. We can also easily verify our claim that triggered star formation occurs on a shorter timescale than the internal enrichment scenario. The most important part of this project will involve a detailed study of how the metals from the supernova remnant mix into the cloud, which will be the determining factor that sets the metallicity of the second-generation stars. If metals do not efficiently mix into the cloud, this could potentially explain how the the low metallicity, CEMP-no stars in Group III formed, and allow us to estimate the properties of their progenitors. If metals mix in more efficiently, this could potentially explain the origin of some of the CEMP-no stars in Group II. Either way, if the conditions for triggered star formation can be achieved in the early universe, \textbf{these simulations will help explain the existence of the CEMP-no stars.}

We will track the gravitational collapse -- or lack thereof -- and determine when star formation is likely to occur by performing various analyses in post-processing. The simplest way to approximately determine when the cloud becomes gravitationally unstable is to measure the ratio of the enclosed mass to the Bonner-Ebert (BE) mass. It is important to use the BE mass in this case because the hot supernova remnant will provide an external pressure on the cool molecular cloud. This pressure acts to decrease the critical mass necessary for gravitational collapse. Another way to determine if stars will form is by directly computing the energy balance of the cloud, $E = (E_{kin} + E_{int}) / (E_{g} + E_{P})$, where $E_{kin}$ is the kinetic energy, $E_{int}$ is the internal energy, $E_{g}$ is the gravitational potential energy, and $E_{P}$ is the energy associated with the surface pressure on the cloud. If E < 1, the cloud is unstable and gravitational collapse will occur.

Multiple studies have shown that triggered star formation via cloud crushing is possible under certain conditions, though none have specifically investigated the transfer of metals into the star forming region (\cite{Melioli2006}, \cite{Leao2009}). We can however, gain useful insight from these existing simulations. \cite{Melioli2006} analytically determined when triggered star formation can occur in the parameter space of supernova radius and cloud density. They confirmed their results with hydrodynamical simulations. We will use these results as a guide for constructing our parameter space, but also test cases which would, according to their model, not induce star formation. Not only will this test the robustness and reproducibility of their results, it may reveal important differences that could arise from different simulation codes, grid resolutions, cooling models, and explosion energies.

With each of our triggered star formation simulations, we will perform a detailed analysis of the metal transfer from the supernova remnant into the cloud. We will investigate how the distribution of metallicity vs density and radius evolves over time. Once the mixing timescale becomes longer than the free fall time of the cloud, metals will no longer efficiently mix into the core, and thus, this will determine the resultant metallicity of the subsequent stars. \cite{...} We will also perform a theoretical estimate of the metal mixing by computing the fraction of metals from the supernova remnant that make it into the cloud, taking into account the timescales for collapse and mixing. 


\begin{figure}
  \caption{Diagram of the triggered star formation scenario.}
    \includegraphics[width=0.5\textwidth]{figures/tsf2}
\end{figure}

Our computational cosmology group has access to the the PACE high-performance computing cluster at Georgia Tech, with 8 dedicated nodes, each with 28 cores, for our use only. We also have access to shared queues with 1000s of cores. These resources will be available to run our simulations and perform the necessary analyses. For easy inspection and analysis of the simulation data, we will use the YT toolkit, which works exceptionally well with Enzo datasets \citep{Turk2011}.

Together, the two independent projects put forth in this proposal will address the question of how we can relate the properties of the second generation stars, with the first generation, Population III stars. Additionally, this work has the power to explain the different formation channels and conditions that lead to the origin of Group II and Group III CEMP-no stars. On a more fundamental level, tracking the metal proliferation in a cosmological simulation and investigating triggered star formation in the early universe, will provide a deep connection between the metal-free universe, and the one we see today, enriched by metals from the first stars. \textbf{Both of these projects are in direct alignment with NASA's Science Mission Directive for the Astrophysics Research Program.} It is clear that we share the same goals of searching for the first stars to understand their formation and the origins of the chemical composition of the universe. Broadly speaking, since the first metals in the universe formed in the first stars, this research will bring us one step closer to understanding one of NASA's most fundamental questions, how did we get here?



\bibliography{proposal}
\end{document}
