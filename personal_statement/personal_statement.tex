\documentclass[letterpaper, 12pt]{article}

\usepackage{natbib}
\usepackage{hyperref}
\usepackage{graphicx}
\usepackage{amsmath}
\usepackage[dvipsnames]{xcolor}
\usepackage[format=plain,justification=RaggedRight, labelfont=bf]{caption}
\usepackage[style=plain,floatrowsep=qquad]{floatrow}
\usepackage{sectsty}
\usepackage[compact]{titlesec}
\usepackage{aas_macros}
\usepackage{amsmath}
\usepackage[charter]{mathdesign}
\usepackage[margin=1in]{geometry}
\usepackage{fancyhdr}
\usepackage{float}
\thispagestyle{fancyplain}
%\fancyhf{}
\rhead{Corey Brummel-Smith}
\lhead{Personal Statement}

\bibliographystyle{apj}

%%%%%%%%%%%%%%%%%%%%%%%%%%%%%%%%%%%%%%%%%%%%%%%%%%%%%%%%%%%%%%%%%%%%%%%%%%%%%%%%
%% fix a bug in bibtex: only the year in author-year should be a link
%%
\RequirePackage{etoolbox}
\makeatletter

% Patch case where name and year have no delimiter
\patchcmd{\NAT@citex}
  {\@citea\NAT@hyper@{\NAT@nmfmt{\NAT@nm}\NAT@date}}
  {\@citea\NAT@nmfmt{\NAT@nm}\NAT@hyper@{\NAT@date}}
  {}% Do nothing if patch works
  {}% Do nothing if patch fails

% Patch case where name and year have basic delimiter
\patchcmd{\NAT@citex}
  {\@citea\NAT@hyper@{%
     \NAT@nmfmt{\NAT@nm}%
     \hyper@natlinkbreak{\NAT@aysep\NAT@spacechar}{\@citeb\@extra@b@citeb}%
     \NAT@date}}
  {\@citea\NAT@nmfmt{\NAT@nm}%
   \NAT@aysep\NAT@spacechar%
   \NAT@hyper@{\NAT@date}}
  {}% Do nothing if patch works
  {}% Do nothing if patch fails

% Patch case where name and year are separated by a prenote
\patchcmd{\NAT@citex}
  {\@citea\NAT@hyper@{%
     \NAT@nmfmt{\NAT@nm}%
     \hyper@natlinkbreak{\NAT@spacechar\NAT@@open\if*#1*\else#1\NAT@spacechar\fi}%
       {\@citeb\@extra@b@citeb}%
     \NAT@date}}
  {\@citea\NAT@nmfmt{\NAT@nm}%
   \NAT@spacechar\NAT@@open\if*#1*\else#1\NAT@spacechar\fi%
   \NAT@hyper@{\NAT@date}}
  {}% Do nothing if patch works
  {}% Do nothing if patch fails

\makeatother
%% end bugfix
%%%%%%%%%%%%%%%%%%%%%%%%%%%%%%%%%%%%%%%%%%%%%%%%%%%%%%%%%%%%%%%%%%%%%%%%%%%%%%%%


%\newcommand{\HRule}{\rule{\linewidth}{0.5mm}}
%\newcommand{\Hrule}{\rule{\linewidth}{0.3mm}}

%\makeatletter% since there's an at-sign (@) in the command name
%\renewcommand{\@maketitle}{%
%  \parindent=0pt% don't indent paragraphs in the title block
%  \centering
%  {\Large \bfseries\textsc{\@title}}
%  \HRule\par%
%  \textit{\@author \hfill \@date}
%  \par
%}
%\makeatother% resets the meaning of the at-sign (@)
%
%\title{Personal Statement}
%\author{Corey Brummel-Smith}
%\date{}

\begin{document}
  %\maketitle

In undergrad, I became possessed with a curiosity to understand the nature of the universe on astronomical scales. Through studying physics and computer science, I became increasingly interested in computational physics simulations, and their power to investigate complex problems in astrophysics, such as star formation and the evolution of the universe as a whole. This set me on a path to pursue my goal of obtaining a Ph.D. in physics, become a professor, and conduct computational astrophysics research at a university or institute. 

My drive to learn about numerical simulations led me to my first research position at Florida State University (FSU), where I was introduced to Enzo: An Adaptive Mesh Refinement Code for Astrophysics. Using Enzo simulations, I investigated the magnetic field structure in turbulent molecular clouds to explain how magnetic fields can modulate and suppress star formation. Another key component of this research was studying how magnetic field structure affects the polarization of foreground dust emission, which is important for understanding the polarization of the cosmic microwave background and the cosmology of our universe \citep{Clark2015}. The Center for Undergraduate Research and Academic Engagement at FSU, awarded me with a research grant and invited me to present my work at the President’s Showcase of Undergraduate Research Excellence, which helped hone my ability to speak about physics and astronomy with a non-technical audience.

Next, I searched for graduate schools that would allow me to dive deeper into my quest to become a computational astrophysicist. This led me to work with Dr. John Wise in the Computational Cosmology Group at the Center for Relativistic Astrophysics at Georgia Tech. My research began by investigating what appeared to be a triggered star formation event in one of his previous Enzo simulations. I discovered a couple of major problems in the simulation. To fix the bugs, I needed to learn about the inner workings of Enzo in greater detail. This understanding is important, not only to fix problems as they arise but also for designing new simulations, like the ones in this proposal. I have attended two Enzo developer workshops where I have contributed to the codebase, reviewed new code modifications, and helped teach new users how to use and edit the code. I am now a contributing author on the most recent Enzo publication \citep{ENZO2019_JOSS}.

In 2018, I was selected along with 16 distinguished graduate students from around the world to attend the Kavli Summer Program in Astrophysics. My project involved creating mock x-ray images of simulated galaxy clusters. By comparing our mock observations, and observations of real clusters, we were able to test the effectiveness of our observational techniques in understanding feedback from active galactic nuclei (AGN). Most importantly, this work has led to an ongoing collaboration. In the following year, one of my mentor's collaborators requested my help in preparing a White Paper for The Decadal Survey on Astronomy and Astrophysics. By using different telescope instrument models and exposure times for my mock observations, we demonstrated how a next-generation telescope with an effective area larger than the Chandra observatory, would allow us to probe the physics of AGN feedback on smaller scales than ever before. Specifically, we showed future telescopes will be able to resolve shock waves and other perturbations that would be hidden in Chandra images with a realistic exposure time \citep{WP_Ruszkowski2019}. 

These experiences have strengthened my drive to become a professor and continue researching unsolved problems in astronomy, such as, what were the properties of the first stars and galaxies, and how can we learn about them today? Through graduate school and research experience, I've developed the professional and technical skills necessary to achieve these goals.

\bibliography{personal_statement}

  
\end{document}