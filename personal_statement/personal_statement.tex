\documentclass[letterpaper, 12pt]{article}

\usepackage{natbib}
\usepackage{hyperref}
\usepackage{graphicx}
\usepackage{amsmath}
\usepackage[dvipsnames]{xcolor}
\usepackage[format=plain,justification=RaggedRight, labelfont=bf]{caption}
\usepackage[style=plain,floatrowsep=qquad]{floatrow}
\usepackage{sectsty}
\usepackage[compact]{titlesec}
\usepackage{aas_macros}
\usepackage{amsmath}
\usepackage[charter]{mathdesign}
\usepackage[margin=1in]{geometry}
\usepackage{fancyhdr}
\usepackage{float}
\thispagestyle{fancyplain}
%\fancyhf{}
\rhead{Corey Brummel-Smith}
\lhead{Personal Statement}

\bibliographystyle{apj}

%%%%%%%%%%%%%%%%%%%%%%%%%%%%%%%%%%%%%%%%%%%%%%%%%%%%%%%%%%%%%%%%%%%%%%%%%%%%%%%%
%% fix a bug in bibtex: only the year in author-year should be a link
%%
\RequirePackage{etoolbox}
\makeatletter

% Patch case where name and year have no delimiter
\patchcmd{\NAT@citex}
  {\@citea\NAT@hyper@{\NAT@nmfmt{\NAT@nm}\NAT@date}}
  {\@citea\NAT@nmfmt{\NAT@nm}\NAT@hyper@{\NAT@date}}
  {}% Do nothing if patch works
  {}% Do nothing if patch fails

% Patch case where name and year have basic delimiter
\patchcmd{\NAT@citex}
  {\@citea\NAT@hyper@{%
     \NAT@nmfmt{\NAT@nm}%
     \hyper@natlinkbreak{\NAT@aysep\NAT@spacechar}{\@citeb\@extra@b@citeb}%
     \NAT@date}}
  {\@citea\NAT@nmfmt{\NAT@nm}%
   \NAT@aysep\NAT@spacechar%
   \NAT@hyper@{\NAT@date}}
  {}% Do nothing if patch works
  {}% Do nothing if patch fails

% Patch case where name and year are separated by a prenote
\patchcmd{\NAT@citex}
  {\@citea\NAT@hyper@{%
     \NAT@nmfmt{\NAT@nm}%
     \hyper@natlinkbreak{\NAT@spacechar\NAT@@open\if*#1*\else#1\NAT@spacechar\fi}%
       {\@citeb\@extra@b@citeb}%
     \NAT@date}}
  {\@citea\NAT@nmfmt{\NAT@nm}%
   \NAT@spacechar\NAT@@open\if*#1*\else#1\NAT@spacechar\fi%
   \NAT@hyper@{\NAT@date}}
  {}% Do nothing if patch works
  {}% Do nothing if patch fails

\makeatother
%% end bugfix
%%%%%%%%%%%%%%%%%%%%%%%%%%%%%%%%%%%%%%%%%%%%%%%%%%%%%%%%%%%%%%%%%%%%%%%%%%%%%%%%


%\newcommand{\HRule}{\rule{\linewidth}{0.5mm}}
%\newcommand{\Hrule}{\rule{\linewidth}{0.3mm}}

%\makeatletter% since there's an at-sign (@) in the command name
%\renewcommand{\@maketitle}{%
%  \parindent=0pt% don't indent paragraphs in the title block
%  \centering
%  {\Large \bfseries\textsc{\@title}}
%  \HRule\par%
%  \textit{\@author \hfill \@date}
%  \par
%}
%\makeatother% resets the meaning of the at-sign (@)
%
%\title{Personal Statement}
%\author{Corey Brummel-Smith}
%\date{}

\begin{document}
  %\maketitle

From a young age, I have been enamored of the vast beauty of outer space. But it was when I first saw the image of the Hubble Deep Field that I knew my life’s work was to study the cosmos. As I studied physics throughout undergrad I became possessed with a curiosity to understand the nature of the universe on astronomical scales. Through studying physics and computer science, I became increasingly interested in computational physics simulations, and their power to investigate the complex, non-linear, problems in modern astrophysics such as star formation and the evolution of the universe as a whole. This set me on a path to pursue my goal of obtaining a Ph.D. in physics, and take the necessary steps to become a professor and conduct research at a university.

My drive to learn about numerical simulations led me to my first research position with Dr. David Collins at Florida State University (FSU), where I was introduced to Enzo: An Adaptive Mesh Refinement Code for Astrophysics. Using Enzo simulations, I investigated the magnetic field structure in turbulent molecular clouds to explain how magnetic fields can modulate and suppress star formation. Another key component of this research was studying how magnetic field structure affects the polarization of foreground dust emission, which is important for understanding the polarization of the cosmic microwave background and the cosmology of our universe \citep{Clark2015}. This research helped me develop the technical skills necessary to succeed as a researcher and computational physicist. For this research, the Center for Undergraduate Research and Academic Engagement at FSU, awarded me with a grant and invited me to present my work at the President’s Showcase of Undergraduate Research Excellence. This gave me the opportunity to hone my ability to speak about physics and astronomy with a non-technical audience; a skill which will benefit me in the future as I strive toward my goal of becoming a professor and science communicator.

At the end of my undergraduate career, I searched for graduate schools that would allow me to dive deeper into my quest to become a computational astrophysicist. This led me to working with Dr. John Wise in the Computational Cosmology group at the Center for Relativistic Astrophysics at Georgia Tech. My research with John began by investigating what appeared to be a triggered star formation event in one of his previous Enzo simulations. I discovered a couple of major problems in the simulations. To fix the bugs, it was necessary for me to learn about the inner workings of Enzo in much greater detail. Over the last two years, I have learned an incredible amount about the Enzo code; from the details of interpolation and flux correction, to how to choose optimal parameters for certain simulations. Knowledge of these details is important, not only to fix problems as they arise, but also to know how to design new simulations, such as the ones in this proposal. I have attended two Enzo developer workshops where I have contributed to the codebase, reviewed new code modifications, and helped teach new users how to use and edit the code. I am now a contributing author on the most recent Enzo publication, which documents recent changes to the code \citep{ENZO2019_JOSS}.

In the summer of 2018, I applied to the Kavli Summer Program in Astrophysics (KSPA), hosted at the Center for Computational Astrophysics (CCA). I was selected to attend this summer school along with 16 other distinguished graduate students from all around the world. With my mentors, Daisuke Nagai, Yuan Li, and Irina Zhuravleva, we worked on project where I created and analyzed mock x-ray images of simulated galaxy clusters. By comparing mock observations of simulated clusters, and real observations of existing ones, we were able to test the effectiveness of our observational techniques in understanding feedback physics from active galactic nuclei (AGN). This unique project helped bridge a gap between simulations and observations. Most importantly, this work has led to an ongoing collaboration, and we have now extended this project far beyond the scope of what we were able to achieve at the KSPA.
In the following year, because of my work at the KSPA, one of Daisuke’s collaborators, requested my help in preparing a White Paper for The Decadal Survey on Astronomy and Astrophysics. I created mock x-ray images of AGN feedback in a simulated galaxy cluster using different telescope instrument models. With these images, we demonstrated how a next-generation telescope with an effective area larger than the Chandra observatory, would allow us to probe the physics of AGN feedback on smaller scales than ever before. Specifically, we showed we would be able to resolve shock waves and other perturbations that would otherwise be hidden in Chandra images with a realistic exposure time \citep{WP_Ruszkowski2019}. 

In the winter of 2019, I traveled to the Kavli Institute for the Physics and Mathematics of the Universe in Tokyo Japan to attend the conference "Stellar Archeology as a Time Machine for the first stars." There I presented my research on metal-transport from a supernova into a nearby molecular cloud found in one of Dr. Wise's simulations of the formation of the first stars. This research is what sparked my idea to study triggered star formation in a systematic and controlled way. Not only was this a great opportunity to share my research with others, but it also allowed me to meet many experts in the field of stellar archeology and the first stars and continue to develop a professional network. At the conference, I learned a lot of valuable information about the state of the field that has helped guide the development of my research plan for this proposal.

All Ph.D students enrolled in the School of Physics at Georgia Tech are required to be a teaching assistant (TA) for the undergraduate physics courses in their first year. For many, this is a tedious distraction from coursework and research, but for me this was a eye-opening and fun endeavor. I taught recitation sections where students would solve context-rich physics problems in small groups. My job was to help students when they had troubles and didn't understand concepts taught in their lecture. I did this by asking thought provoking questions to help guide them without giving away answers. I never knew how much I would enjoy teaching and sparking the student's interest in physics. There is something about spreading the knowledge that I love with others that brings me joy and satisfaction. This is one of the primary reasons why I want to become a physics or astronomy professor in the future.

These experiences have strengthened my drive to become a professor and continue researching unsolved problems in astronomy, such as, what were the properties of the first stars and galaxies, and how can we learn about them today? Through graduate schooling and research experience, I’ve developed the professional and technical skills necessary to succeed as a computational physicist and achieve these goals.

\noindent{\textbf{Graduate study timeline:} \\
I began my graduate studies in Fall 2017. As is the for most physics grad students on the Ph.D track at Georgia Tech, I took the core upper level physics courses my first year. In Spring 2018, I joined Dr. Wise's Computational Cosmology group and also received my M.S. in physics from Georgia Tech. The next two years I continued research with Dr. Wise while taking one to two astrophysics courses per semester. I will select a Ph.D. committee and give my thesis proposal at the end of my third year (Spring 2020). Upon passing the thesis proposal exam, I will be admitted as a Ph.D. candidate. I will then continue research until I complete my dissertation. On average, students in the School of Physics take approximately six years to complete their Ph.D. Using this an estimate means I expect to graduate in Spring 2023.}

\bibliography{personal_statement}

  
\end{document}